\chapter{Laporan Harian dan Mingguan}
Laporan harian dan mingguan ini dibuat untuk pengarsipan kegiatan sivitas akademika selama berada dilingkugan \textit{Informatics Research Center} (IRC).

Tujuan pembuatan laporan harian dan mingguan sebagai berikut :
\begin{enumerate}
\item Sebagai informasi individu selama berada dilingkungan IRC
\item Sebagai tolak ukur keaktifan individu selama berada dilingkungan IRC
\item Sebagai penilaian individu terhadap aktivitas yang dilakukan dilingkungan IRC
\end{enumerate}

\section{Contoh Laporan Harian dan Mingguan dengan Tabel}
\begin{table}[H]
\caption{Laporan Harian Tanggal 25 Februari 2018}
\label{tab:lh250219}
\begin{tabular}{|l|l|l|}
\hline
\textbf{No} & \multicolumn{1}{c|}{\textbf{Kategori}} & \multicolumn{1}{c|}{\textbf{Keterangan}} \\ \hline
1 & Dedikasi & - \\ \hline
2 & Produktifitas & \begin{tabular}[c]{@{}l@{}}a. Membuat repositori Modul Praktikum kelas Pemrograman III\\     Kelas 2A, 2B, dan 2C di organisasi pemograman-iii.\end{tabular} \\
 &  & b. Mengintal python, pip, anaconda, dan jupiter dari Udemy. \\
 &  & \begin{tabular}[c]{@{}l@{}}c. Melakukan merge pull request di pemograman-iii/praktikum\_2a,\\    dari \#1, \#3, \#4, \#7, \#8, \#9, \#10, \#12, \#13, dan \#14.\end{tabular} \\ \hline
3 & Integritas & able to merge/has no conflict \\ \hline
4 & Disiplin & Jam Datang : 07.07 WIB \\
 &  & Jam Pulang : 16.20 WIB \\ \hline
5 & Loyalitas & \begin{tabular}[c]{@{}l@{}}Membersihkan dan merapihkan meja kerja, dan mengecek AC\\ di pagi dan sore hari.\end{tabular} \\ \hline
\end{tabular}
\end{table}

% Please add the following required packages to your document preamble:
% \usepackage[normalem]{ulem}
% \useunder{\uline}{\ul}{}
\begin{table}[H]
\caption{Laporan Harian Tanggal 26 Februari 2018}
\label{tab:lh260219}
\begin{tabular}{|l|l|l|}
\hline
\textbf{No} & \multicolumn{1}{c|}{\textbf{Kategori}} & \multicolumn{1}{c|}{\textbf{Keterangan}} \\ \hline
1 & Dedikasi & - \\ \hline
2 & Produktifitas & a. Membuat repositori Modul Praktikum kelas Kecerdasan Buatan. \\
 &  & b. Mendata dan menilai tugas 1 kelas 2A-Pemrograman III. \\
 &  & \begin{tabular}[c]{@{}l@{}}c. Membimbing Fadila, Lusia, dan Rahmi Roza kelas 3C untuk \\     tugas Kecerdasan Buatan.\end{tabular} \\ \hline
3 & Integritas & able to merge/has no conflict \\ \hline
4 & Disiplin & Jam Datang : 08.30 WIB \\
 &  & Jam Pulang : 16.30 WIB \\ \hline
5 & Loyalitas & \begin{tabular}[c]{@{}l@{}}Membersihkan dan merapihkan meja kerja, dan mengecek AC\\ di pagi dan sore hari.\end{tabular} \\ \hline
\end{tabular}
\end{table}

\begin{table}[H]
\caption{Laporan Harian Tanggal 27 Februari 2019}
\label{tab:lh270219}
\begin{tabular}{|l|l|l|}
\hline
\textbf{No} & \multicolumn{1}{c|}{\textbf{Kategori}} & \multicolumn{1}{c|}{\textbf{Keterangan}} \\ \hline
1 & Dedikasi & bukuinformatika/flask \#10 \\ \hline
2 & Produktifitas & a.Mendata dan menilai tugas kelas 3C-Kecerdasan Buatan. \\
 &  & b.Membimbing anak kelas 3C untuk tugas Kecerdasan Buatan. \\
 &  & c. Mengkoordinasi tugas kontribusi anak kelas. \\ \hline
3 & Integritas & able to merge/has no conflict \\ \hline
4 & Disiplin & Jam Datang : 08.50 WIB \\
 &  & Jam Pulang : 17.30 WIB \\ \hline
5 & Loyalitas & \begin{tabular}[c]{@{}l@{}}Membersihkan dan merapihkan meja kerja, dan mengecek AC\\ di pagi dan sore hari.\end{tabular} \\ \hline
\end{tabular}
\end{table}

\begin{table}[H]
\caption{Laporan Harian Tanggal 28 Februari 2019}
\label{tab:lh280219}
\begin{tabular}{|l|l|l|}
\hline
\textbf{No} & \multicolumn{1}{c|}{\textbf{Kategori}} & \multicolumn{1}{c|}{\textbf{Keterangan}} \\ \hline
1 & Dedikasi & BukuInformatika/flask \#12 \\ \hline
2 & Produktifitas & \begin{tabular}[c]{@{}l@{}}a. Evaluasi mingguan dan sosialisasi format baru penilaian peserta\\     internship II di IRC dan Prodi.\end{tabular} \\
 &  & \begin{tabular}[c]{@{}l@{}}b. Memeriksa lembar jawaban UAS GIS kelas 3C dan Arkom\\     kelas 1C.\end{tabular} \\
 &  & c. Menginput nilai UAS kelas 1C, 3A, 3B, dan 3C di google docs. \\
 &  & \begin{tabular}[c]{@{}l@{}}d. Mengkoordinir mahasiswa untuk tugas kontribusi pembuatan cover,\\     pencetakan, sampai pendistribusian buku di Grup Whatsapp.\end{tabular} \\
 &  & \begin{tabular}[c]{@{}l@{}}e. Memeriksa tugas kecerdasan buatan kelas 3C serta menginput nilai\\     ke google docs atas nama Fadila, Lusia Violita Aprilian, dan Rahmi.\end{tabular} \\ \hline
3 & Integritas & able to merge/has no conflict \\ \hline
4 & Disiplin & Jam Datang : 07.55 WIB \\
 &  & Jam Pulang : 17.00 WIB \\ \hline
5 & Loyalitas & \begin{tabular}[c]{@{}l@{}}Menyapu, membersihkan dan merapihkan meja, mencuci gelas nomor 6,\\ dan mengecek AC di pagi dan sore hari.\end{tabular} \\ \hline
\end{tabular}
\end{table}

\begin{table}[H]
\caption{Laporan Harian Tanggal 1 Maret 2019}
\label{tab:lh010319}
\begin{tabular}{|l|l|l|}
\hline
\textbf{No} & \multicolumn{1}{c|}{\textbf{Kategori}} & \multicolumn{1}{c|}{\textbf{Keterangan}} \\ \hline
1 & Dedikasi &  \\ \hline
2 & Produktifitas & \begin{tabular}[c]{@{}l@{}}a. Mengerjakan Soal Toefl Pre-Test 2 (Computer Based)\\     Skor Toefl: 55\end{tabular} \\
 &  & b. Mendata skor toefl peserta Internship II di IRC. \\
 &  & c. Memeriksa pull request di laporanirc/2019. \\ \hline
3 & Integritas & able to merge/has no conflict \\ \hline
4 & Disiplin & Jam Datang : 07.39 WIB \\
 &  & Jam Pulang : 14.20 WIB \\ \hline
5 & Loyalitas & \begin{tabular}[c]{@{}l@{}}Menyapu, membersihkan dan merapihkan meja, membeli\\ sabun dan spons cuci piring, dan mengecek AC di pagi\\ dan sore hari.\end{tabular} \\ \hline
\end{tabular}
\end{table}

\section{Penilaian Mingguan (Tabel)}
Penilaian mingguan ini dilakukan setiap hari Jum'at, penilain ini akumulasi dari total laporan harian.

\begin{table}[H]
\centering
\caption{Nilai Minggu ke-1}
\label{tab:nm01}
\begin{tabular}{|c|c|c|}
\hline
\textbf{No} & \textbf{Kategori} & \textbf{Poin} \\ \hline
1 & Dedikasi & 1 \\ \hline
2 & Produktifitas & 3 \\ \hline
3 & Integritas & 3 \\ \hline
4 & Disiplin & 3 \\ \hline
5 & Loyalitas & 3 \\ \hline
 & \textbf{Total Poin} & 13 \\ \hline
\end{tabular}
\end{table}
