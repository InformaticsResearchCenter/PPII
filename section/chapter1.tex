\chapter{Tugas Pokok}

Tugas Pokok merupakan pekerjaan yang harus atau wajib dikerjakan selama berada dilingkungan \textit{Informatics Research Center} (IRC). Pelaporan tugas ini wajib dilakukan setiap satu hari sekali dengan menyertakan parameter yang sudah ditentukan yaitu DEDIKASI, PRODUKTIFITAS, INTEGRITAS, DISIPLIN, LOYALITAS, serta KREATIFITAS dan INISIATIF. Untuk penilaian tugas pokok yang sudah di kerjakan satu minggu sebelumnya dilakukan kembali berapa score yang di dapat setiap satu minggu sekali dengan mengakumulasikan dengan poin yang sudah di tentukan terlebih dahulu jumlah pekerjaan per-parameter selama 1 minggu. Poin minimal untuk penilaian mingguan yaitu sebanyak 13 poin. Jika jumlah poin tidak mencapai poin yang sudah di tentukan minimal 13, maka mahasiswa harus menambah pekan untuk mengganti nilai yang kurang tersebut.

\section{Dedikasi}

Menurut KBBI (Kamus besar Bahasa Indonesia), Dedikasi bisa diartikan sebagai suatu pengorbanan tenaga, pikiran, dan waktu yang kita miliki demi keberhasilan suatu usaha atau tujuan yang mulia. Dalam kata lain dedikasi juga bisa diartikan sebagai pengabdian terhadap suatu pekerjaan kita atau tempat kita belajar. Pengabdian atau dedikasi yang bisa dilakukan mahasiswa D4 Teknik Informatika yaitu dengan melakukan pembuatan ataupun pembaharuan modul ajaran yang sudah ada di github atau bisa langsung di lihat di link ini \textbf{\textit{https://github.com/bukuinformatika}}.

Adapun parameter penilaian dedikasi dapat dilihat pada tabel \ref{table:nilaidedikasi}.

\begin{table}[H]
\caption{Penilaian Dedikasi}
\centering
\begin{tabular}{|c|c|c|c|}
\hline
\textbf{No.}&\textbf{Label}&\textbf{Nilai}&\textbf{Keterangan}\\
\hline
1.&TINGGI&3&Full commit selama 5 hari kerja\\
\hline
2.&SEDANG&2&Commit selama 4 hari kerja\\
\hline
3.&RENDAH&1&Commit kurang dari 4 hari kerja\\
\hline
\end{tabular}
\label{table:nilaidedikasi}
\end{table}

\section{Produktifitas}
PProduktivitas merupakan suatu sikap individu dimana kita harus bisa memanejemen waktu dengan baik supaya kita dapat menghasilkan suatu karya atau menghasilkan suatu hal yang bermanfaat bagi diri kita sendiri. Contoh dari pengimplementasian produktivitas misalnya seperti disaat kita melakukan suatu pekerjaan harian pada suatu tempat atau ketika kita sedang melakukan studi, kita harus dapat kita pertanggung jawabkan semua kegiatan yang kita lakukan dalam pekerjaan tersebut.

Adapun parameter penilaian produktifitas dapat dilihat pada tabel \ref{table:nilaiproduktifitas}.

\begin{table}[H]
\caption{Penilaian Produktifitas}
\centering
\begin{tabular}{|c|c|c|c|}
\hline
\textbf{No.}&\textbf{Label}&\textbf{Nilai}&\textbf{Keterangan}\\
\hline
1.&TINGGI&3&Mengerjakan 5 pekerjaan harian\\
\hline
2.&SEDANG&2&Mengerjakan 4 pekerjaan harian\\
\hline
3.&RENDAH&1&Mengerjakan kurang dari 4 pekerjaan harian\\
\hline
\end{tabular}
\label{table:nilaiproduktifitas}
\end{table}

\section{Integritas}
Integritas merupakan salah satu atribut terpenting/kunci yang harus dimiliki seseorang. Dimana integritas ini dapat juga diartikan sebagai syarat untuk kita dimana kita harus/wajib menyelesaikan suatu pekerjaan yang di berikan dengan baik dan tepat waktu, dan didalam integritas ini kita di tuntut untuk meminimalisir kesalah kita dalam suatu pekerjaan.

Adapun parameter penilaian integritas dapat dilihat pada tabel \ref{table:nilaiintegritas}.

\begin{table}[H]
\caption{Penilaian Integritas}
\centering
\begin{tabular}{|c|c|c|c|}
\hline
\textbf{No.}&\textbf{Label}&\textbf{Nilai}&\textbf{Keterangan}\\
\hline
1.&TINGGI&3&Tidak ada penolakan pull request\\
\hline
2.&SEDANG&2&Ada 1 penolakan pull request\\
\hline
3.&RENDAH&1&Ada lebih dari 1 penolakan pull request\\
\hline
\end{tabular}
\label{table:nilaiintegritas}
\end{table}

Catatan:
\begin{itemize}
\item Selesaikan konflik terlebih dahulu, untuk menghindari penolakan saat pull request.
\end{itemize}

\section{Disiplin}
Disiplin merupakan persyaratan yang wajib di ikuti, perasaan taat dan patuh terhadap nilai-nilai yang ada disiplin juga dapat di artika sebagai tanggung jawab terhadap pekerjaan dan waktu yang di berikan. Dengan kata lain disiplin adalah patuh terhadap peraturan yang berlaku
atau tunduk pada pengawasan.

Adapun parameter penilaian disiplin dapat dilihat pada tabel \ref{table:nilaidisiplin}.

\begin{table}[H]
\caption{Penilaian Disiplin}
\centering
\begin{tabular}{|c|c|c|c|}
\hline
\textbf{No.}&\textbf{Label}&\textbf{Nilai}&\textbf{Keterangan}\\
\hline
1.&TINGGI&3&Datang pulang sesuai jadwal selama 5 hari kerja\\
\hline
2.&SEDANG&2&Ada 1 hari terlambat ataupun tidak masuk\\
\hline
3.&RENDAH&1&Ada lebih dari 1 hari terlambat ataupun tidak masuk\\
\hline
\end{tabular}
\label{table:nilaidisiplin}
\end{table}

\section{Loyalitas}
Loyalitas adalah dimana diri kita sendiri dapat melakukan pekerjaan yang menurut kita dapat kita kerjakan tanpa ada paksaan sedikit pun
dengan tulus dan tanpa meminta belas kasihan dari orang lain atau loyalitas dapat di artikan melakukan pekerjaan tanpa meminta imbalan sedikit pun terhadap lingkungan kerja kita.
Contoh dari loyalitas diantaranya:
\begin{enumerate}
\item Menjaga area kerja tetap bersih dan bebas dari debu;
\item Menjaga kerapihan dan kenyaman area kerja;
\item Memperbaiki perangkat kerja yang rusak, dll.
\end{enumerate}
Adapun parameter penilaian loyalitas dapat dilihat pada tabel \ref{table:nilailoyalitas}.

\begin{table}[H]
\caption{Penilaian Loyalitas}
\centering
\begin{tabular}{|c|c|c|c|}
\hline
\textbf{No.}&\textbf{Label}&\textbf{Nilai}&\textbf{Keterangan}\\
\hline
1.&TINGGI&3&Area kerja bersih tanpa debu selama 5 hari kerja\\
\hline
2.&SEDANG&2&Ada 1 hari area kerja tidak bersih\\
\hline
3.&RENDAH&1&Ada lebih dari 1 hari area kerja tidak bersih\\
\hline
\end{tabular}
\label{table:nilailoyalitas}
\end{table}

\section{Kreatif dan Inisiatif}

Adapun parameter penilaian kreatif dan inisiatif dapat dilihat pada tabel ~\ref{table:nilaikreatifinisiatif}.

\begin{table}[H]
\caption{Penilaian Kreatif dan Inisiatif}
\centering
\begin{tabular}{|c|c|c|c|}
\hline
\textbf{No.}&\textbf{Label}&\textbf{Nilai}&\textbf{Keterangan}\\
\hline
1.&TINGGI&3&Menjadi tentor dengan jumlah peserta minimal 10 orang\\
\hline
2.&SEDANG&2&Menjadi tentor dengan jumlah peserta minimal 5 orang\\
\hline
3.&RENDAH&1&Menjadi tentor dengan jumlah peserta kurang dari 5 orang\\
\hline
\end{tabular}
\label{table:nilaikreatifinisiatif}
\end{table}

Catatan:
\begin{enumerate}
\item Kegiatan ini dilaksanakan minimal 1 kali selama Internship atau kegiatan berlangsung, dengan catatan target poin terpenuhi.
\item Target poin yang dicapai minimal 3.
\item Jika poin tidak memenuhi target, maka silahkan untuk mengadakan kegiatan lagi sampai jumlah poin minimal terpenuhi.
\end{enumerate} 